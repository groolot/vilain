\documentclass[a4paper,titlepage,oneside]{article}

\usepackage{../../vilain}

\addbibresource{../../vilain.bib}
\usepackage[%
	pdftex,%
	bookmarks=false,%
	bookmarksopen=false,%
	bookmarksnumbered=false,%
	pdfborder=0,%
	pdfmenubar=true,%
	pdftoolbar=true,%
	pdfpagelabels=true,%
	pdftitle={GUI decisioning},%
	pdfauthor={Ronan LEGARDINIER, Gr\'egory DAVID},%
	pdfsubject={vilain, gui},%
	pdfpagelayout={OneColumn},%
	plainpages=true,%
	pdfstartview=FitH%
]{hyperref}

%%%%%%%%%%%%%%%%%%%%%%%%%%%%%%%%%%%%%%%%%%
% Easily include image w/wo box around
% usage:
%     \image{path/to/file.png}{width=13cm}{This is the caption}{fig:ThiIsTheReferenceLabel}
%     \imageBoxed{path/to/file.png}{width=13cm}{This is the caption}{fig:ThiIsTheReferenceLabel}
%
% after what you can reference the image by:
%     \ref{fig:ThiIsTheReferenceLabel}
%
%%%%%%%%%%%%%%%%%%%%%%%%%%%%%%%%%%%%%%%%%%

%%%%%%%%%%%%%%%%%%%%%%%%%%%%%%%%%%%%%%%%%%
% Easily include numbered tabular
% usage:
%     \tableau{|l|p{4cm}|c|}{Caption}{tab:Label}{
%     \hline
%     A & B & C \\ \hline
%     etc.
%     }
%
% after what you can reference the tabular by:
%     \ref{tab:Label}
%
%%%%%%%%%%%%%%%%%%%%%%%%%%%%%%%%%%%%%%%%%%

\author{Ronan \bsc{Legardinier} (2014) \and Grégory \bsc{David} (2010 -- \the\year)}
\title{\vilain{} -- Graphical User Interface library selection}
\date{\today{}}

\begin{document}
\maketitle{}
\tableofcontents{}

\section{Context}
% This is a LateX file used to be included inside other documents
% to assure coherence between documentations.
% This file describe the vilain project developpment context. 
\vilain{} is a standalone program, usefull for lighting project based
on video-projection. \vilain{} is the \texttt{C++11} porting effort of
the VSL (Virtual Stage Light) PureData abstractions prototype
(Cf. \url{http://sourceforge.net/projects/virtualslight/}).

It's aimed to provide a simple way to do:
\begin{itemize}
    \item 2D video mapping on 3D surface (modifying input content
    geometry),
    \item webcam tracking,
    \item VJing (triggering, filtering, compositing flat layers),
    \item video lighting.
\end{itemize}

It is designed to run on GNU/Linux platform with a better integration
with the i3-wm window manager. Raspberry Pi hardware is covered by the
main development and integration is done frequently.

Other platforms (like Mac OS X and Windows) are not planned yet, as not so
important to me at the moment.


\section{Needs, goal}
\vilain{} GUI require graphics elements like grouping, value
selectors, toggles, tabs, lists, sortable lists, 2D pads, text input
area, radios button, checkboxes, undraggable panels and a monitoring
space like \figurename \ref{fig:GuiDesign}.

\image{data/GUIDesign.jpg}{width=13cm}{\vilain{} Gui
  design}{fig:GuiDesign}

\section{Constraints}
\vilain{} GUI require graphical elements like:
\begin{itemize}
    \item Grouping,
    \item Tabs,
    \item Sliders,
    \item Toggles,
    \item Text inputs,
    \item Buttons,
    \item Undraggable panels,
    \item Lists,
    \item Sortable lists,
    \item Tables,
    \item Radio buttons,
    \item Checkboxes.
\end{itemize}

\section{Possible solutions}
After several tests with some \OF{} Graphical User Interface addons
libraries: \texttt{ofxControlPanel}, \texttt{ofxTweakBar},
\texttt{ofxUI} (Cf. \figurename \ref{fig:ofxUIexample}) and
\texttt{ofxGui} (shipped with \OF{}, Cf. \figurename
\ref{fig:ofxGuiexample}), we have made a multi criterions comparison.

\image{data/ofxGui.jpg}{width=10cm}{\texttt{ofxGUI} basic example,
  from the \OF{} shipped addons
  (\texttt{openframeworks/examples/gui/guiExample})}{fig:ofxGuiexample}

\image{data/ofxUI.jpg}{width=11cm}{\texttt{ofxUI} example, from the
  \texttt{ofxUI/example-AllWidgets}}{fig:ofxUIexample}

\section{Comparison \& Selection}
The \tablename \ref{tab:Comparison} is a comparison of two libraries
within the constraints listed above.

\tableau{|p{3.5cm}|c|c|c|c|c|}{\OF{} GUI addons libraries comparison}{tab:Comparison} { \hline
  \textbf{Requirements} & Coefficient & \texttt{ofxGui} & \texttt{ofxUI} & \texttt{ofxControlPanel} & \texttt{ofxTweakBar} \\
  \hline
  \multicolumn{6}{|c|}{Source code} \\
  \hline
  GNU/Linux friendly & 5 & + & + & + & - \\
  Reliable & 5 & + & + & - & + \\
  Documented & 2 & - & - & - & - \\
  Low memory foot print & 1 & + & + & + & + \\
  Responsive & 3 & + & + & - & + \\
  \hline
  Total(1) & 16 & 14 & 14 & 6 & 9 \\
  \hline
  \multicolumn{6}{|c|}{Required components} \\
  \hline
  Grouping & 3 & - & +(Matrix) & - & - \\
  Tabs & 3 & - & + & + & - \\
  Sliders & 3 & + & + & + & + \\
  Toggles & 3 & + & + & + & + \\
  Text inputs & 3 & - & + & - & + \\
  Buttons & 3 & + & + & + & + \\
  Undraggable panels & 3 & - & + & + & - \\
  Lists & 3 & - & - & + & + \\
  Radio buttons & 3 & - & + & - & - \\
  Checkboxes & 3 & + & + & + & + \\
  Tables & 3 & - & - & - & - \\
  \hline
  Total(2) & 33 & 12 & 27 & 21 & 18 \\
  \hline
  \multicolumn{6}{|c|}{Optional components} \\
  \hline
  Vertical radio buttons & 2 & - & + & - & - \\
  Sortable lists & 2 & - & + & - & - \\
  Color sliders & 1 & + & - & - & - \\
  \hline
  Total(3) & 5 & 1 & 4 & 0 & 0 \\
  \hline \hline
  \textbf{Total} (1)+(2)+(3) & 54 & 27 & \textbf{45} & 27 & 27 \\
  \hline }

\texttt{ofxUI} seems to be better than \texttt{ofxGui} due to the
number of graphics elements it offers, particularly those needed for
the project.\\

\texttt{ofxUI} offers really a lot of user-friendly and
operational, reliable and well-developed elements.  \texttt{ofxUI}
allows the possibility to pass by reference the variables of object in
graphical elements, allowing an easier development of \texttt{OSC}
interface.

\section{Implementation example}
\subsection{Example}
The \figurename \ref{fig:GUIVilainExample} has been token from
\vilain{} project.
\image{data/GUIVilain.jpg}{width=12cm}{\texttt{ofxUI} example
  according to \vilain{} design}{fig:GUIVilainExample}


\subsection{Implementation with \texttt{ofxUI}}
\begin{itemize}
    \item \textbf{Global:} canvas have to be set up on top of
    everything else
    \item \textbf{Canvas:} canvas are stored in a vector, the order
    they are stored the order they are shown
    \item \textbf{Widgets:} widgets are stored in a per canvas vector,
    the order they are stored the order they are shown

    \item \textbf{Instanciate a canvas:}
    \lstinline[language=C++]{ofxUICanvas* myCanvas = new ofxUICanvas(float _xPosition, float _yPosition, float _width, float _height);}
	
    \item \textbf{Associate widgets to a canvas:} \subitem
    \lstlistingname{} \vref{lst:widgetExample_1} shows an example of
    how to add a label widget to a canvas, directly through the
    already instanciate canvas \subitem \lstlistingname{}
    \vref{lst:widgetExample_2} shows an example of how to add a label
    widget to a canvas, first of instantiating the widget and then
    adding it to the canvas.  \subitem \lstlistingname{}
    \vref{lst:widgetExample_3} shows an example of how to add a label
    widget to a canvas, first of instantiating the widget and then set
    the canvas as parent of it

    \item \textbf{Event management:} \subitem Events are catched by a
    listener, that call a function at each event in a canvas - See
    \lstlistingname{} \vref{lst:eventListener} \subitem At each event
    in a canvas, the listener call a function with as parameter the
    name of the clicked widget - See \lstlistingname{}
    \ref{lst:eventManagement}
\end{itemize}

\subsection*{Examples}
\begin{lstlisting}[language=C++, caption={\texttt{ofxUIWidget} first way implementation}, label={lst:widgetExample_1}]
ofxUICanvas* myCanvas = new ofxUICanvas();
myCanvas->addLabel("My label");
\end{lstlisting}
			
\begin{lstlisting}[language=C++, caption={\texttt{ofxUIWidget} second way implementation}, label={lst:widgetExample_2}]
ofxUICanvas * myCanvas = new ofxUICanvas();
ofxUILabel * myLabel = new ofxUILabel("My label", OFX_UI_FONT_MEDIUM);
myCanvas->addWidget(myLabel);
\end{lstlisting}
		
\begin{lstlisting}[language=C++, caption={\texttt{ofxUIWidget} third way implementation}, label={lst:widgetExample_3}]
ofxUICanvas * myCanvas = new ofxUICanvas();
ofxUILabel * myLabel = new ofxUILabel("My label", OFX_UI_FONT_MEDIUM);
myLabel->setParent(myCanvas);
\end{lstlisting}
		
\begin{lstlisting}[language=C++, caption={\texttt{ofxUI} event listener}, label={lst:eventListener}]
ofAddListener(myCanvas->newGUIEvent, this, &vilainApp::mainUI_Event);
\end{lstlisting}
		
\begin{lstlisting}[language=C++, caption={\texttt{ofxUI} event listener}, label={lst:eventManagement}]
void vilainApp::mainUI_Event(ofxUIEventArgs &e)
{
    string eventName = e.getName();
    if(eventName == "My label")
    {
        //...
    }
}
\end{lstlisting}


% References
%\newpage
\printbibheading
\printbibliography[nottype=online,check=notonline,heading=subbibliography,title={Bibliography}]
\printbibliography[check=online,heading=subbibliography,title={Webography}]
\nocite{openframeworks,ofxUI,REZAALI,OF_TUTORIALS_ofauckland,ofBook}

\end{document}