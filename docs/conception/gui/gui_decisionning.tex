\documentclass[a4paper,titlepage,oneside]{article}

\usepackage[french,english]{babel}
\usepackage[utf8]{inputenc}
\usepackage[T1]{fontenc}
\usepackage[pdftex]{graphicx}
\usepackage[usenames]{color}
\usepackage[backend=bibtex8,backref=true]{biblatex}
    \addbibresource{vilain.bib}
\usepackage{varioref} % References croisees qui vont bien
\usepackage[%
	pdftex,%
	bookmarks=false,%
	bookmarksopen=false,%
	bookmarksnumbered=false,%
	pdfborder=0,%
	pdfmenubar=true,%
	pdftoolbar=true,%
	pdfpagelabels=true,%
	pdftitle={GUI decisioning},%
	pdfauthor={Ronan LEGARDINIER, Gr\'egory DAVID},%
	pdfsubject={vilain, gui},%
	pdfpagelayout={OneColumn},%
	plainpages=true,%
	pdfstartview=FitH%
]{hyperref}

%%%%%%%%%%%%%%%%%%%%%%%%%%%%%%%%%%%%%%%%%%
% Define macro to easily include image w/wo box around
% usage:
%     \image{path/to/file.png}{width=13cm}{This is the caption}{fig:ThiIsTheReferenceLabel}
%     \imageBoxed{path/to/file.png}{width=13cm}{This is the caption}{fig:ThiIsTheReferenceLabel}
%
% after what you can reference the image by:
%     \ref{fig:ThiIsTheReferenceLabel}
%
%%%%%%%%%%%%%%%%%%%%%%%%%%%%%%%%%%%%%%%%%%
\newcommand{\image}[5][\textwidth]{%
% #1 = minipage width
% #2 = file path
% #3 = includegraphics options
% #4 = caption text
% #5 = reference label
    \begin{figure}[h]
        \centering
            \begin{minipage}[c]{#1}
                \centering
                \includegraphics[#3]{#2}
                \caption{#4}
                \label{#5}
            \end{minipage}
    \end{figure}
}
\newcommand{\imageBoxed}[5][\textwidth]{%
% #1 = minipage width
% #2 = file path
% #3 = includegraphics options
% #4 = caption text
% #5 = reference label
    \begin{figure}[h]
        \centering
        \fbox{
            \begin{minipage}[c]{#1}
                \centering
                \includegraphics[#3]{#2}
                \caption{#4}
                \label{#5}
            \end{minipage}
        }
    \end{figure}
}

%%%%%%%%%%%%%%%%%%%%%%%%%%%%%%%%%%%%%%%%%%
% Define macro to easily include numbered tabular
% usage:
%     \tableau{|l|p{4cm}|c|}{Caption}{tab:Label}{
%     \hline
%     A & B & C \\ \hline
%     etc.
%     }
%
% after what you can reference the tabular by:
%     \ref{tab:Label}
%
%%%%%%%%%%%%%%%%%%%%%%%%%%%%%%%%%%%%%%%%%%
\newcommand{\tableau}[4]{
% #1 = structure string
% #2 = caption
% #3 = label
% #4 = content
	\begin{table}[h]
        \centering
        \begin{tabular}{#1}
            #4
        \end{tabular}
        \caption{#2}
        \label{#3}
    \end{table}
}

\newcommand{\vilain}{\textbf{vilain::}}

\author{Ronan \bsc{Legardinier}, Grégory \bsc{David}}
\title{\vilain{} -- Graphical User Interface library selection}
\date{\today{}}

\begin{document}
\maketitle{}
\tableofcontents{}
\newpage

\section{Context}
\vilain{} project. Graphic library selection to GUI development.

\section{Needs}
\vilain{} GUI require graphics elements like value selectors, toggles, tabs, lists, sortable lists, 2D pads, text input area, radios button, checkboxes, undraggable panels and a monitoring space like \figurename \vref{fig:GuiDesign}

\image{data/GUIDesign.jpg}{width=13cm}{\vilain{} Gui design}{fig:GuiDesign}

\newpage
\section{Constraints}
\vilain{} GUI require elements offer by several libraries like :\\
- Tabs\\
- Sliders\\
- Color sliders\\
- Toggles\\
- Text inputs\\
- Buttons\\
- Undraggable panels\\
- Lists\\
- Vertical radio buttons\\
- Sortable lists\\
- Tables
- Radio buttons\\
- Checkboxes\\

\section{Possible solutions}
We have two solution of library:
\subsection{ofxGui}
The \figurename \vref{fig:ofxGuiexample} has been token from the \texttt{openframeworks/examples/gui/guiExample}.
\image{data/ofxGui.jpg}{width=13cm}{Gui example with ofxGUI}{fig:ofxGuiexample}

\newpage
\subsection{ofxUi}
The \figurename \vref{fig:ofxUIexample} has been token from the \texttt{ofxUI/example-AllWidgets}.
\image{data/ofxUI.jpg}{width=12cm}{Gui example with ofxUI}{fig:ofxUIexample}

\newpage
\section{Comparison / Selection}
The \tablename \vref{tab:Comparison} is a comparison of two libraries within the constraints listed above.\\
\tableau{|p{4cm}|c|c|}{Comparison between ofxGUI and ofxUI}{tab:Comparison}
{
	\hline
	Requirements & ofxGUI & ofxUI\\
	\hline
	Tabs & - & + \\
	Sliders & + & + \\
	Color sliders & + & - \\
	Toggles & + & + \\
	Text inputs & - & + \\
	Buttons & + & + \\
	Undraggable panels & - & + \\
	Lists & - & - \\
	Vertical radio buttons & - & + \\
	Sortable lists & - & + \\
	Radio buttons & - & + \\
	Checkboxes & - & + \\
	\hline
}

ofxUI seems to be better than ofxUI due to the number of graphics
elements it offers, particularly those needed for the project.

\newpage
\section{Implementation example}
The \figurename \vref{fig:TabBarExample} has been token from the \texttt{ofxUI/example-TabBar}.
\image{data/TabBar.jpg}{width=5cm}{Tabs bar example with ofxUI}{fig:TabBarExample}

The \figurename \vref{fig:GUIVilainExample} has been token from \vilain{} project.
\image{data/GUIVilain.jpg}{width=13cm}{\vilain{} Gui example with ofxUI}{fig:GUIVilainExample}
\end{document}