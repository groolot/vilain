\documentclass[a4paper,titlepage,oneside]{article}

\usepackage{../../vilain}

\addbibresource{../../vilain.bib}
\usepackage[%
	pdftex,%
	bookmarks=false,%
	bookmarksopen=false,%
	bookmarksnumbered=false,%
	pdfborder=0,%
	pdfmenubar=true,%
	pdftoolbar=true,%
	pdfpagelabels=true,%
	pdftitle={Multi-Windowing},%
	pdfauthor={Gr\'egory DAVID},%
	pdfsubject={vilain, gui, multi-windowing},%
	pdfpagelayout={OneColumn},%
	plainpages=true,%
	pdfstartview=FitH%
]{hyperref}

%%%%%%%%%%%%%%%%%%%%%%%%%%%%%%%%%%%%%%%%%%
% Easily include image w/wo box around
% usage:
%     \image{path/to/file.png}{width=13cm}{This is the caption}{fig:ThiIsTheReferenceLabel}
%     \imageBoxed{path/to/file.png}{width=13cm}{This is the caption}{fig:ThiIsTheReferenceLabel}
%
% after what you can reference the image by:
%     \ref{fig:ThiIsTheReferenceLabel}
%
%%%%%%%%%%%%%%%%%%%%%%%%%%%%%%%%%%%%%%%%%%

%%%%%%%%%%%%%%%%%%%%%%%%%%%%%%%%%%%%%%%%%%
% Easily include numbered tabular
% usage:
%     \tableau{|l|p{4cm}|c|}{Caption}{tab:Label}{
%     \hline
%     A & B & C \\ \hline
%     etc.
%     }
%
% after what you can reference the tabular by:
%     \ref{tab:Label}
%
%%%%%%%%%%%%%%%%%%%%%%%%%%%%%%%%%%%%%%%%%%

\author{Grégory \bsc{David}}
\title{\vilain{} -- Multi-Windowing}
\date{\today{}}

\newcommand{\ofxFenster}{\texttt{ofxFenster}}

\begin{document}
\maketitle{}
\tableofcontents{}
\newpage

\section{Context}
% This is a LateX file used to be included inside other documents
% to assure coherence between documentations.
% This file describe the vilain project developpment context. 
\vilain{} is a standalone program, usefull for lighting project based
on video-projection. \vilain{} is the \texttt{C++11} porting effort of
the VSL (Virtual Stage Light) PureData abstractions prototype
(Cf. \url{http://sourceforge.net/projects/virtualslight/}).

It's aimed to provide a simple way to do:
\begin{itemize}
    \item 2D video mapping on 3D surface (modifying input content
    geometry),
    \item webcam tracking,
    \item VJing (triggering, filtering, compositing flat layers),
    \item video lighting.
\end{itemize}

It is designed to run on GNU/Linux platform with a better integration
with the i3-wm window manager. Raspberry Pi hardware is covered by the
main development and integration is done frequently.

Other platforms (like Mac OS X and Windows) are not planned yet, as not so
important to me at the moment.


\section{Needs, goal}
As the projector could not be orthogonal to projection plane, and even
the surface could not be planar, the goal is to have a dual screen
setup with two independent images (see \figurename{}
\vref{fig:multiScreenSchema}):
\begin{enumerate}
    \item [User side:] Object control, object layering, application
    settings.
    \item [Projector side:] Rendered image (in performance mode), same
    objects modification image as the user side (in editing mode).
\end{enumerate}

\image{data/multi_windowing_schema.png}{width=13cm}{Multi screen
  schema}{fig:multiScreenSchema}

\section{Constraints}
\begin{itemize}
    \item Two separate windows,
    \item Duplicate rendered surface on both,
    \item Control elements do not have to be drawn on the projector
    side,
    \item Events are managed in the user side window,
    \item Control can be done remotely through OSC.
\end{itemize}

\section{Possible solutions}
Due to GLFW usage inside \OF{}, we do have only one solution:
\texttt{ofxFenster} addon.

Another solution is to use an external application, such as \emph{Open
  Stage Control} from Jean-Emmanuel \textsc{Doucet}
\cite{OpenStageControl}.

\section{Implementation}
\ofxFenster{} is an addon developped by Philip \bsc{Whitfield}: see
\url{https://github.com/underdoeg}.

The easiest way is to render the shared content into a Frame Buffer Object (FBO) and then draw it where it has to be.

FBO is supllied with ease of use by \OF{} through \lstinline[basicstyle=\footnotesize\ttfamily]{ofFbo} class.

TODO: explain how to :)

\newpage
% References \newpage
\printbibheading
\printbibliography[nottype=online,check=notonline,heading=subbibliography,title={Bibliography}]
\printbibliography[check=online,heading=subbibliography,title={Webography}]
\nocite{openframeworks,ofxUI,REZAALI,OF_TUTORIALS_ofauckland,ofBook}

\end{document}