\documentclass[a4paper,oneside]{article}

\usepackage{../vilain}
\usepackage{pdflscape}
\usepackage{mathtools}
\usepackage{amssymb}
\usepackage{array}

\addbibresource{../vilain.bib}
\usepackage[%
	pdftex,%
	bookmarks=false,%
	bookmarksopen=false,%
	bookmarksnumbered=false,%
	pdfmenubar=true,%
	pdftoolbar=true,%
	pdfpagelabels=true,%
	pdftitle={Visual identity},%
	pdfauthor={Gr\'egory DAVID},%
	pdfsubject={vilain, visual identity},%
	pdfpagelayout={OneColumn},%
	pdfstartview=FitH%
]{hyperref}

\definecolor{vilainRed}{cmyk}{0,0.78,0.78,0.4}

\newcommand{\proposition}[4][\textwidth]{%
% #1 = minipage width
% #2 = file path
% #3 = caption text
% #4 = reference label
    \begin{figure}[h]
        \centering
            \begin{minipage}[c]{#1}
                \centering
                \includegraphics[height=0.3cm]{#2}
                \vline{}
                \includegraphics[height=0.5cm]{#2}
                \vline{}
                \includegraphics[height=1cm]{#2}
                \vline{}
                \includegraphics[height=2cm]{#2}
                \caption{#3}
                \label{#4}
            \end{minipage}
    \end{figure}
}

\author{Grégory \bsc{David}}
\title{\vilain{} -- Visual identity}
\date{\today{}}

\begin{document}
\maketitle{}
\tableofcontents{}

\section{Text}
\subsection{Font}
\subsubsection{In baseline and logotype}
\paragraph{Name} eurosteal
\paragraph{Provider}
\url{http://fontstruct.com/fontstructions/show/986192}
\paragraph{License} The FontStruction “Eurosteal” by Michał “Neoqueto”
Nowak is licensed under a Creative Commons Attribution Share Alike
license (\url{http://creativecommons.org/licenses/by-sa/3.0/}).

\subsection{Colors}
\paragraph{Black}
\subparagraph{CMYK} 35, 35, 35, 100

\subparagraph{Example} \textcolor{black}{This is a simple text in
  black, \textbf{and this one is bold}}

\paragraph{Dark purple on logo and text}
\subparagraph{CMYK} 0, 78, 78, 40

\subparagraph{Example} \textcolor{vilainRed}{This is a simple text in
  purple, \textbf{and this one is bold}}

\subsubsection{In plain text}
References to the project name have to follow this simples constraints.

\paragraph{Syntax} \texttt{\vilain{}}
\paragraph{Style} \vilain{} In bold case when possible, else as is.

\section{Baseline}
The \tablename{} \vref{tab:Usages} summarizes how to use the
baselines: colored (see \figurename{} \vref{fig:baselineColor}),
grayscale (see \figurename{} \vref{fig:baselineGrayscale}) or black \&
white (see \figurename{} \vref{fig:baselineBlackWhite}) ones.

\proposition{baseline_color_cmyk.pdf}{Colored
  baseline}{fig:baselineColor}
\proposition{baseline_grayscale_cmyk.pdf}{Grayscale
  baseline}{fig:baselineGrayscale}
\proposition{baseline_black-and-white_cmyk.pdf}{Black\&White
  baseline}{fig:baselineBlackWhite}

\cleardoublepage{}
\section{Logotype}
The \tablename{} \vref{tab:Usages} summarizes how to use the
logotypes: colored (see \figurename{} \vref{fig:logoColor}),
grayscale (see \figurename{} \vref{fig:logoGrayscale}) or black \&
white (see \figurename{} \vref{fig:logoBlackWhite}) ones.

\proposition{logo_color_cmyk.pdf}{Colored logotype}{fig:logoColor}
\proposition{logo_grayscale_cmyk.pdf}{Grayscale
  logotype}{fig:logoGrayscale}
\proposition{logo_black-and-white_cmyk.pdf}{Black\&White
  logotype}{fig:logoBlackWhite}

\cleardoublepage{}
\section{Usages}
\tableauSized{l*{6}{>{$}c<{$}}}{Baseline and Logotype usages}{tab:Usages}{
  \firsthline
  Context & \text{\figurename{} \ref{fig:baselineColor}} & \text{\figurename{} \ref{fig:baselineGrayscale}} & \text{\figurename{} \ref{fig:baselineBlackWhite}} & \text{\figurename{} \ref{fig:logoColor}} & \text{\figurename{} \ref{fig:logoGrayscale}} & \text{\figurename{} \ref{fig:logoBlackWhite}} \\
  \hline \hline
  Paper color print & \checkmark \\
  
  \lasthline
}{\textwidth}

\cleardoublepage{}
\section{Forbidens}
\proposition{baseline_black-and-white_negative_cmyk.pdf}{Negative Black\&White
  baseline is \textbf{FORBIDEN}}{fig:baselineNegativeBlackWhite}
\proposition{logo_black-and-white_negative_cmyk.pdf}{Negative Black\&White
  logotype is \textbf{FORBIDEN}}{fig:logoNegativeBlackWhite}



\end{document}
